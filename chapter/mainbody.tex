% -*- coding: utf-8 -*-

\chapter{引言}
这是引言。这是引言。这是引言。这是引言。这是引言。这是引言。这是引言。这是引言。这是引言。这是引言。这是引言。这是引言。这是引言。这是引言。这是引言。这是引言。这是引言。这是引言。这是引言。这是引言。这是引言。这是引言。这是引言。这是引言。这是引言。这是引言。这是引言。这是引言。这是引言。这是引言。这是引言。这是引言。这是引言。这是\cite{test}引言。这是引言。这是引言。这是引言。这是引言。这是引言。这是引言。这是引言。这是引言。这是引言。这是引言。这是引言。这是引言。这是引言。这是引言。这是引言。这是引言。这是引言。这是引言。这是引言。这是引言。这是引言。


\chapter{模板简介}
本模板使用多文件结构,参与\LaTeX 编译的主要文件放在chapter子文件夹里,需要插入的图片文件等请放在images文件夹里。

本模板的标题,正文字体,页边距,目录,参考文献,行间距等主要论文格式已经设定完毕。
\section{封面}
封面样式已经设定完成,你只需要将封面对应信息填写在命令\verb|\NKTsetup|中即可。这一命令对应参数已经在main.tex文件中给出,请按需求填写。使用命令\verb|\NKTsetup|的同时本科生毕业论文的声明也将同时生成
\section{摘要与关键词}
摘要和关键词的相关信息请在abstract.tex中填写,摘要和关键词都需要中英文,中文摘要请在\verb|abstractcn|环境中填写,英文摘要请在\verb|abstract|环境中填写;中文关键词请在\verb|keywordscn|环境中填写,英文关键词请在\verb|keywords|环境中填写。
\section{目录}
目录由\verb|\tableofcontents|生成,这一命令已在main.tex中给出。在编译生成pdf的时候,为了生成正确的目录,可能需要编译2次。点击目录对应的标题可直接跳转至对应页面。
\section{正文}
正文内容请在mainbody.tex中填写。正文设有3个层级,章使用\verb|chapter|环境生成,节使用\verb|section|环境生成,小节使用\verb|subsection|环境生成。正文内容包括数学环境正常编辑即可。
\subsection{字体}
ctex宏集中提供了\verb|\songti|({\songti 宋体}),\verb|\heiti|({\heiti 黑体}),\verb|\fangsong|({\fangsong 仿宋}),\verb|\kaishu|({\kaishu 楷体})四种基本字体,如有需要请直接使用。另外本模板提供了\verb|\jiacu|({\jiacu 加粗})命令,在保持字体的基础上对其加粗。

如需要使用罗马数字,本模板提供了命令\verb|\Romannum|和\verb|\romannum|来分别生成大小写罗马数字。如\Romannum{2},\romannum{4}。
\subsection{字号}
字号请使用命令\verb|\zihao|来改变。如需要3号字,使用命令如{\zihao{3}这是三号字}。如需要对应的小号字,使用其相反数作为参数,如小3号字为{\zihao{-3}这是小三号字}。
\subsection{脚注}
脚注使用\verb|\footnote|生成,如\footnote{这是一个脚注}。
\subsection{图表}
本模板使用图表不采用浮动体形式,也即不使用环境\verb|figure|与\verb|table|,需要插入图片和表格时请使用\verb|center|环境。图片的标题使用\verb|\figurecaption|来生成,位于图片之下;表格的标题使用\verb|tablecaption|来生成,位于表格之上。如
\begin{center}
	\includegraphics[height=2cm]{nankaidaxue.jpg}
	\figurecaption{这是标题}
\end{center}
如果想要同一行插入多张图片,共用一个标题,那么只需像下面这么写,可以控制中间需要空多宽
\begin{center}
	\includegraphics[height=2cm]{nankaidaxue.jpg}\hspace{0.8cm}
	\includegraphics[height=2cm]{nankaidaxue.jpg}
	\figurecaption{这是标题}
\end{center}
特别注意论文格式要求中,表格一律采用三线表形式。如
\begin{center}
	\tablecaption{我是表格}
	\begin{tabular}{ccc}
		\hline
		条目一 & 条目二 & 条目三 \\ \hline
		内容一 &   1   &    2   \\
		内容二 &   2   &    2   \\
		内容三 &   3   &    3   \\
		内容四 &   4   &    4   \\ \hline
	\end{tabular}
\end{center}
\section{参考文献}
参考文献使用bib文件生成。如有需要的参考文献,你可以在相应的论文网站上导出bib格式,再将内容复制到文件nkthesis.bib中。引用文献一般使用\verb|\cite|命令,如\cite{zhaoliu},同时引用多个文件如\cite{zhaoliu,sunqian,chenjing,wuqiang}。同样,参考文献也需要多次编译才能正确生成,如果第一遍编译没法生成参考文献,请多再编译几次。参考文献对应的tex文件为references.tex,你不需要修改这一文件。

根据物理科学学院2018级毕业论文督察组的审核结果,光学组要求参考文献中,英文部分人名的首字母大写,其余小写。因此本模板在编写参考文献的时候,输入的时候大小写是什么,编译产生的大小写就是什么。比如你的输入为\verb|author={KoSeKi}|,那参考文献的作者部分输出就是KoSeKi。

注意:正文引用什么文献,后面的参考文献就按照正文引用的顺序自动生成;如果正文不引用某篇文献,那么后面参考文献将不显示这一篇。
\section{附录}
附录请在appendix.tex中填写。
\section{致谢}
致谢请在acknowledgements.tex中填写。
\section{个人简历}
个人简历请在resume.tex中填写。考虑到此命令不常用,默认注释掉。如有需求请在main.tex中取消对\verb|\include{chapter/resume}|命令的注释。
\begin{equation}
	2+6=8
\end{equation}
\begin{equation}
	1+1=2\hbar
\end{equation}